\documentclass[pdftex12pt, a4paper]{article}
\usepackage[pdftex]{graphicx}
\usepackage{amsmath}
\usepackage{boxedminipage}
\usepackage{hyperref}
\usepackage{fullpage}

\begin{document}
\begin{titlepage}

\title{OSS Iteratie 1}
\author{Michiel Huygen\\Tom Jacobs\\Victor Jacobs\\Jeroen Wygaerts}

\maketitle
\thispagestyle{empty}

\end{titlepage}

\newpage

\tableofcontents

\newpage


\section*{JSettlers}

The goal of the first part of this project for the course ``Ontwerp van softwaresystemen'' is to analyse an opensource-project, named JSettlers. 
This project is an implementation of the board game ``Settlers of Catan'' and is written in java.

\section{Introduction}
The analysis of the software was done on 2 parallel tracks. 
One track was to use code analysis tools to find problematic elements in the sourcecode. 
The other track used javadocs generated from the project's source. 
The javadoc gives a general idea of the distribution of responsiblities within the project.

The software seems to work without any problems; distributed setups cause no problems.

Our initial impression of the project is that very little time was spent on the actual software design. 
The documentation is extensive, although not 100\% complete. 
This seems necessary since the code is extremely incohesive. 
It seems an impressive feat that no bugs were encountered during our short tests of the software, considering the complexity of the code. 


\section{Ontwerpdocumentatie}

\newpage

\section{Evaluatie ontwerp}

Maaanyy godclasses

The design structures that are present in the project seem to have been added while coding. 
There are some inheritance hierarchies present but it is not used for polymorfysm 

\section{Patronen}

\newpage

\section{Analysetools}

\subsection{Coverlipse}

\newpage

\section{Testen}

\newpage

\section{Besluit}

Hello 


\newpage

\section{Projectbeheer}

\end{document}